\documentclass[10pt,xcolor=dvipsnames,aspectratio=169]{beamer}\usepackage[]{graphicx}\usepackage[]{color}
% maxwidth is the original width if it is less than linewidth
% otherwise use linewidth (to make sure the graphics do not exceed the margin)
\makeatletter
\def\maxwidth{ %
  \ifdim\Gin@nat@width>\linewidth
    \linewidth
  \else
    \Gin@nat@width
  \fi
}
\makeatother

\definecolor{fgcolor}{rgb}{0.345, 0.345, 0.345}
\newcommand{\hlnum}[1]{\textcolor[rgb]{0.686,0.059,0.569}{#1}}%
\newcommand{\hlstr}[1]{\textcolor[rgb]{0.192,0.494,0.8}{#1}}%
\newcommand{\hlcom}[1]{\textcolor[rgb]{0.678,0.584,0.686}{\textit{#1}}}%
\newcommand{\hlopt}[1]{\textcolor[rgb]{0,0,0}{#1}}%
\newcommand{\hlstd}[1]{\textcolor[rgb]{0.345,0.345,0.345}{#1}}%
\newcommand{\hlkwa}[1]{\textcolor[rgb]{0.161,0.373,0.58}{\textbf{#1}}}%
\newcommand{\hlkwb}[1]{\textcolor[rgb]{0.69,0.353,0.396}{#1}}%
\newcommand{\hlkwc}[1]{\textcolor[rgb]{0.333,0.667,0.333}{#1}}%
\newcommand{\hlkwd}[1]{\textcolor[rgb]{0.737,0.353,0.396}{\textbf{#1}}}%
\let\hlipl\hlkwb

\usepackage{framed}
\makeatletter
\newenvironment{kframe}{%
 \def\at@end@of@kframe{}%
 \ifinner\ifhmode%
  \def\at@end@of@kframe{\end{minipage}}%
  \begin{minipage}{\columnwidth}%
 \fi\fi%
 \def\FrameCommand##1{\hskip\@totalleftmargin \hskip-\fboxsep
 \colorbox{shadecolor}{##1}\hskip-\fboxsep
     % There is no \\@totalrightmargin, so:
     \hskip-\linewidth \hskip-\@totalleftmargin \hskip\columnwidth}%
 \MakeFramed {\advance\hsize-\width
   \@totalleftmargin\z@ \linewidth\hsize
   \@setminipage}}%
 {\par\unskip\endMakeFramed%
 \at@end@of@kframe}
\makeatother

\definecolor{shadecolor}{rgb}{.97, .97, .97}
\definecolor{messagecolor}{rgb}{0, 0, 0}
\definecolor{warningcolor}{rgb}{1, 0, 1}
\definecolor{errorcolor}{rgb}{1, 0, 0}
\newenvironment{knitrout}{}{} % an empty environment to be redefined in TeX

\usepackage{alltt}


\definecolor{maybebad}{HTML}{d7e519}
\definecolor{likelybad}{HTML}{db981c}
\definecolor{verylikelybad}{HTML}{db3f1c}
\usetheme{metropolis}
\usepackage{appendixnumberbeamer}
\usepackage{epstopdf}
\usepackage{pgfpages}
\input{notes.config}

\usepackage{pgfplots}
\usepgfplotslibrary{dateplot}

\usepackage{booktabs}
\usepackage[scale=2]{ccicons}

% optional metro things
\metroset{numbering=fraction} % page numbering format
\metroset{progressbar=foot} % progress bar in footer
\metroset{sectionpage=none, subsectionpage=none}
\metroset{block=fill}
% \metroset{background=light}

\setsansfont{OpenSans}[
Path = ./fonts/OpenSans/,
Extension = .ttf,
UprightFont     =   *-Regular,
BoldFont        =   *-Bold,
ItalicFont      =   *-Italic,
BoldItalicFont  =   *-BoldItalic
]

\setmonofont{RobotoMono}[
Path = ./fonts/RobotoMono/,
Extension = .ttf,
UprightFont     =   *-Regular,
BoldFont        =   *-Bold,
ItalicFont      =   *-Italic,
BoldItalicFont  =   *-BoldItalic
]

\definecolor{makopurple1}{RGB}{57,39,91}
\definecolor{makopurple2}{RGB}{137,119,173}
\definecolor{makopurple3}{RGB}{29,13,59}
\definecolor{makopurple4}{RGB}{124,99,173}

% \definecolor{makopurple1}{HTML}{3b255b}
\definecolor{makogreen1}{HTML}{0b7a75}
\definecolor{makovanilla1}{HTML}{d7c9aa}
\definecolor{makoumber1}{HTML}{7b2d26}
\definecolor{makowhite1}{HTML}{f0f3f5}

\setbeamercolor{normal text}{fg=black, bg=white}
\setbeamercolor{alerted text}{fg=makopurple4}
\setbeamercolor{example text}{fg=makoumber1}

\setbeamercolor{palette primary}{bg=makopurple1, fg=normal text.bg}
\setbeamercolor{palette secondary}{bg=makogreen1,fg=normal text.bg}
\setbeamercolor{palette tertiary}{bg=makovanilla1,fg=normal text.bg}
\setbeamercolor{palette quaternary}{bg=makoumber1,fg=normal text.bg}
\setbeamercolor{structure}{fg=makopurple1, bg=normal text.bg} % itemize, enumerate, etc
\setbeamercolor{section in toc}{fg=makopurple1, bg=normal text.bg} % TOC sections

% \setbeamercolor{progress bar}{fg=makopurple1}
% \setbeamercolor{title separator}{fg=makopurple1}
% \setbeamercolor{progress bar in head/foot}{fg=makopurple1}
% \setbeamercolor{progress bar in section page}{fg=makopurple1}

% this is a fix for a bug with XeTeX and notes pages
% see: https://tex.stackexchange.com/questions/232168/normal-text-is-invisible-when-using-beamer-with-notes-and-xelatex
\usepackage{ifthen}
\makeatletter
\def\beamer@framenotesbegin{% at beginning of slide
  \ifthenelse{\not \boolean{metropolis@standout}}{%
    \usebeamercolor[fg]{normal text}%
      \gdef\beamer@noteitems{}% 
      \gdef\beamer@notes{}%
      
    }{\gdef\beamer@noteitems{}%
      \gdef\beamer@notes{}%
    }
}
% redfine the notes type to clear things (working around the same bug)
% https://tex.stackexchange.com/questions/28966/change-beamer-notes-list-type-for-in-frame-notes
\def\beamer@setupnote{%
  \gdef\beamer@notesactions{%
    \beamer@outsideframenote{%
      \beamer@atbeginnote%
      {\normalfont%
       \parskip 0.5em%
      \usebeamercolor[fg]{normal text}%
      \beamer@notes%
      \ifx\beamer@noteitems\@empty\else
      \beamer@noteitems%
      \fi%
      
      \par}
      \beamer@atendnote%
    }%
    \gdef\beamer@notesactions{}%
  }
}
\makeatother

\usepackage{xspace}
\usepackage{relsize}

% add tikz and a bunch of tikz foo
\usepackage{tikz}
\usetikzlibrary{shapes,shapes.misc,backgrounds,fit,positioning}
\tikzstyle{every picture}+=[overlay,remember picture]

% add functions to circle parts of slides (e.g., in tables)
\newcommand\marktopleft[1]{\tikz \node (marker-#1-a) at (0,1.5ex) {};}
\newcommand\markbottomright[1]{%
  \tikz{\node (marker-#1-b) at (0,0) {};}
  \tikz[dashed,inner sep=3pt]{\node[violet!75,ultra thick,draw,rounded rectangle,fit=(marker-#1-a.center) (marker-#1-b.center)] {};}}

% create an empty quotetxt so we can reuse it
\newcommand{\quotetxt}{}

% more flexible non-tikz alternative with no dropshadow 
\newlength{\centertxtlen}
\makeatletter
\newcommand\centertext[2]{%
  \setlength{\centertxtlen}{#1}%
  \setlength{\centertxtlen}{0.48\centertxtlen}%
  {\centering
    \fontsize{#1}{2\centertxtlen}\selectfont
    \e{#2}

  }
}

% packages i use in essentially every document
\usepackage{graphicx}
\usepackage{url}
% \usepackage{dcolumn}
% \usepackage{booktabs}
\graphicspath{{resources}{figure}{figures}}
% replace footnotes with symbols instead of numbers
\renewcommand*{\thefootnote}{\fnsymbol{footnote}}
\usepackage{perpage}
\MakePerPage{footnote}

% \hypersetup{colorlinks=true, linkcolor=Black, citecolor=Black, filecolor=makopurple1,
%     urlcolor=Plum, unicode=true}

% create a new \e{} command to make things purple and bold
\newcommand{\e}[1]{\alert{#1}}

% remove the nagivation symbols
\newcommand{\greenscreen}{
\begin{tikzpicture}
  \draw[fill=green] (13.1,5.4) rectangle ++(4/1.35,5/1.35);
\end{tikzpicture}
}

\addtobeamertemplate{footline}{}{\greenscreen}

\setbeamertemplate{navigation symbols}{}

\newcommand{\credit}[1]{%
  \tikz[overlay]{\node at (current page.south west)
    [anchor=south west, yshift=3pt, xshift=2pt]
    {\smaller \smaller {[}#1{]}};}}

\setlength{\parsep}{1em}

\title{The effects of algorithmic flagging on fairness:}
\subtitle{Quasi-experimental evidence from Wikipedia}
\author[TeBlunthuis,Hill,Halfaker]{\textbf{Nathan TeBlunthuis}
  \footnotesize{nathante@uw.edu}\inst{1} \inst{2} \\ \and \textbf{Benjamin Mako Hill} \inst{1} \and \\ \textbf{Aaron Halfaker} \inst{2}}

\institute[UW,WMF]{\textbf{University of Washington\inst{1}} \\
  Department of Communication \\
{\textbf{Wikimedia Foundation}\inst{2}}
}

\date{June 19, 2020}
\IfFileExists{upquote.sty}{\usepackage{upquote}}{}
\begin{document}

% remove some of the space in the itemize to make it quite compact
% \let\olditemize\itemize
% \renewcommand\itemize{\olditemize\itemsep-1pt}

%% SLIDE: Title Slide
\begin{frame}
  \titlepage

  \tikz{\node at (current page.center) [anchor=center, yshift=-.75in, xshift=1.6in]
    {\includegraphics[width=0.25\textwidth]{figures/logo.pdf}};
  }

  % include version control stuff
  % \input{vc}
  % \credit{Revision:\ \VCRevision\  (\VCDateTEX)}
\end{frame}

% \section{How will algorithmic flagging shape fairness of human decision making?}

% \begin{frame}
% \larger
%   \sectionpage

%   \note{This is an example section page.}
% \end{frame}

\begin{frame}{What are we talking about?}

\larger \larger

\emph{Algorithmic predictions} that \emph{route attention} or \emph{provide information} to influence \emph{human decisions} that \emph{affect another person}.

\note{This project is about a kind of sociotechnical system where algorithmic predictions route attention or provide information to influence consequential human decisions
 that affect other people} 
\end{frame}


% \begin{frame}
%   \frametitle{Simplifying a debate: Will algorithms help or hurt fairness?}

% \larger
% \larger
%   \e{Taste-based discrimination:}  \\ \emph{``I don't like people that look like you.''}

%   Algorithmic flagging can be \emph{biased}; \\ reproduce or encode problematic categories or ontologies;

%   \pause \vspace{1em}

%   \e{Statistical discrimination:} \\  \emph{``People that look like you are likely to cause problems.''}

% \end{frame}

% \begin{frame}
% \larger \larger
% \enspace 
% \enspace
% \enspace

% Can algorithmic flagging help by better accounting for more information than human judges and thereby reduce statistical discrimination?

% \end{frame}



\begin{frame}\frametitle{High stakes, legally complicated algorithms}
\larger
\centering
\includegraphics[width=0.35\textwidth]{figures/court.pdf}  \\ \pause
\vspace{-2em}
\includegraphics[width=0.35\textwidth]{figures/job_application.jpg} \hfill \pause
\includegraphics[width=0.35\textwidth]{figures/house.png} \pause
\note{Some applications are high stakes, legally complicated: Should a judge grant bail to a defendant? Should a job applicant get an interview?  Should an applicant get a lease? These applications might be difficult to study for practical and ethical reasons.  }

\end{frame}

\begin{frame}\frametitle{Lower stakes applications in social media moderation}

\includegraphics[width=0.45\textwidth]{figures/wikipedia_logo.png}\hfill
\includegraphics[width=0.45\textwidth]{figures/reddit_logo.png}

\note{Similar kinds of systems are used in different contexts with lower stakes. In online moderation, how people are treated by moderators impacts who has voice in communities, whose views are represented in important knowledge bases, and the ability of moderators to prevent misbehavior, but not individual life courses or systematic oppression.}

\end{frame}

\section{Algorithmic flagging in online content moderation on Wikipedia}

\begin{frame}
  \sectionpage
\end{frame}



\begin{frame}\frametitle{Recent changes filters on Wikipedia}
\vspace{15em}
    \centering
\larger
\begin{tikzpicture}

  \node[anchor=west](flags) at (-7,2.7) {ORES Flags};
%  \node[anchor=west](pagetitle) at (-4.7,2.7) {Page titles};
  \node[anchor=west](userpagelink) at (-2.9,2.7) {Red user profile link};
  \node[anchor=west](unregistered) at (1.4,2.7) {Unregistered editor};


  \begin{scope}
  \node[anchor=south, inner sep=0] (image) at (0,0) {\includegraphics[width=\textwidth]{resources/rcfilters_example_2.png}};
  \draw [-stealth,ultra thick] (flags.220) -- ++(0,-0.4);
%  \draw [-stealth,ultra thick] (pagetitle.south) -- ++(0,-0.4);
  \draw [-stealth,ultra thick] (unregistered.200) -- (1.5,2);
  \draw [-stealth,ultra thick] (userpagelink.280) -- (0.2,1.6);
\end{scope}
\end{tikzpicture}

\raggedright

\note{Flags are triggered by arbitrary thresholds, so we can use a regression discontinuity to estimate the causal effects of being flagged on moderator actions?}

\end{frame}


\begin{frame}\frametitle{Moderation and fairness}

  \larger 
\enspace

  Moderators use \emph{social signals} like \emph{registration} \\ and \emph{experience} to find and \emph{sanction} misbehavior. 

  \enspace

  People displaying such signals might be \e{over-profiled} if moderators focus their attention on them but not on others engaged in similar behavior.

\enspace

  Moderator actions can be considered unfair if they are \e{controversial sanctions} (i.e. violates a meta-norm or a third person objects to the initial sanction).

  \note{When we think about how an algorithmic flagging system shapes fairness it's important to think about the baseline.  How are moderators making decisions without the algorithmic signals? How are they making them with the signals?

    Social signals like registration status and experience are clues that an editor might be misbehaving.  The moderators might use these signals the way that car insurance companies use sex as a sign that certain types of people are more risky than others.  This can lead to over-profiling if they focus their attention and their sanctioning actions on people that display these signals instead of others that might be up to equally bad if not even worse behavior.


    This might lead to unfairness in terms of disproportionate sanctioning or in terms of incorrect sanctioning in terms of meta-norms that tell moderators when they should issue sanctions
  }
  \end{frame}

\begin{frame}\frametitle{Research questions}
\vspace{7em}
\larger

  \e{RQ1:} How will flagging an action change the likelihood an action is sanctioned for over-profiled editors compared to others?

  \e{RQ2:} How will flagging an action by an over-profiled editor change the chances it receives a controversial sanction?

  \e{RQ3:} Within the set of sanctioned actions, how will the effect of flagging an action on controversial sanctions depend on whether contributors are over-profiled?

\end{frame}

\begin{frame}\frametitle{Research questions}
\vspace{7em}
\larger

  \e{RQ1:} How will flagging an action change the likelihood an action is sanctioned for over-profiled editors compared to others?

  \e{RQ2:} How will flagging an action by an over-profiled editor change the chances it receives a controversial sanction?

\textcolor{gray}{\textbf{RQ3:} Within the set of sanctioned actions, how will the effect of flagging an action on controversial sanctions depend on whether contributors are over-profiled?}

\end{frame}

\begin{frame}\frametitle{Research design: regression discontinuity}

\larger

An algorithmic predictor triggers flags when prediction \\
scores an \emph{arbitrary threshold}.

  So we infer the causal effects of flagging on moderation by \\ comparing edits right above the threshold with and  \\ conditioning  on the scores.

\begin{minipage}{0.45\textwidth}
\begin{knitrout}
\definecolor{shadecolor}{rgb}{0.969, 0.969, 0.969}\color{fgcolor}
\includegraphics[width=\textwidth]{figures/knitr-adoption_me_plot-1} 

\end{knitrout}
\end{minipage}
\hfill
\begin{minipage}{0.45\textwidth}
By the way, we use Bayesian models fit with Rstanarm. 

\note{This lets us draw a robust comparison between edits that are flagged and edits that are not flagged}
\end{minipage}

\end{frame}

\begin{frame}\frametitle{Data: Wikimedia History}

\larger


\begin{itemize}
\item Public data of Wikipedia edit history
\item Historical prediction scores maintained by Wikimedia \\ (to be released)
\item Thresholds (\textcolor{maybebad}{maybe damaging}, \textcolor{likelybad}{likely damaging})\\ reconstructed by loading old models and configuration files
\item Stratified sample with up to 23 different language editions of Wikipedia.
\end{itemize}

Unit of analysis: \emph{revision to an article}

\begin{description}

\item[overprofiling:] \emph{registration status} and \emph{redlink User pages}.
\item[sanction:] a revision is undone by a second party.
\item[controversial sanction:] a sanction is undone by a third party.

\end{description}

\note{We didn't find overall effects on sanctioning at the highest (very likely damaging) threshold, so we exclude it.}
\end{frame}

\section{Results}

\begin{frame}
  \sectionpage
  \end{frame}

\begin{frame}\frametitle{RQ1: Flagging and over-profiling: Registration status}

\begin{knitrout}
\definecolor{shadecolor}{rgb}{0.969, 0.969, 0.969}\color{fgcolor}
\includegraphics[width=\maxwidth]{figures/knitr-h1_unreg_me_plot-1} 

\end{knitrout}
\end{frame}


\begin{frame}\frametitle{RQ2: Flagging and fair sanctioning: Unregistered editors}

\larger 

Flagging decreases controversial sanctioning \\ for unregistered editors. 


\includegraphics[width=1\linewidth]{figures/knitr-me_plot_H2_anon-1} 


\end{frame}

% \begin{frame}\frametitle{RQ1: Flagging and over-profiling: Registration status}

%  \larger

% We find quite large effects of flagging on sanctioning. 

% Unflagged actions by unregistered editors \\ are far more likely to be reverted. 

% Once flagged, actions by registered and unregistered editors are reverted at relatively similar rates.

% This is evidence that Wikipedia moderators overprofile based on registration status and that algorithmic flagging reduces this over-profiling. 

% % Being flagged at the ``maybe damaging'' increases the probability of sanctioning from  for unregistered contributors a change of between round((anon.md.proto.above$linpred.lower - anon.md.proto.below$linpred.upper)*100,1) and round((anon.md.proto.above$linpred.upper - anon.md.proto.below$linpred.lower)*100,1)
% % percentage points. 

% % For registered contributors the increase is \emph{larger:} from format.percent(non.anon.md.proto.below$linpred,1) to format.percent(non.anon.md.proto.above$linpred,1), change of between round((non.anon.md.proto.above$linpred.lower - non.anon.md.proto.below$linpred.upper)*100,1) and round((non.anon.md.proto.above$linpred.upper - non.anon.md.proto.below$linpred.lower)*100,1) percentage points.  

% % Similarly, at the ``likely damaging'' threshold we see an increases in sanctioning probability from format.percent(anon.ld.proto.below$linpred,1) to format.percent(anon.ld.proto.above$linpred,1) for unregistered contributors a change of between round((anon.ld.proto.above$linpred.lower - anon.ld.proto.below$linpred.upper)*100,1) and round((anon.ld.proto.above$linpred.upper - anon.ld.proto.below$linpred.lower)*100,1)
% % percentage points. 

% % And for registered contributors the increase is \emph{larger:} from format.percent(non.anon.ld.proto.below$linpred,1) to format.percent(non.anon.ld.proto.above$linpred,1), change of between round((non.anon.ld.proto.above$linpred.lower - non.anon.ld.proto.below$linpred.upper)*100,1) and round((non.anon.ld.proto.above$linpred.upper - non.anon.ld.proto.below$linpred.lower)*100,1) percentage points.  
% \end{frame}

% \begin{frame}\frametitle{Flagging improves fairness for unregistered editors}

% \begin{description}
% \item[Reducing overprofiling:] Once flagged, they were sanctioned \\ at a similar rate to \emph{registered} editors. 
% \item[Improve moderator decisions:] Flagging makes sanctions against unregistered editors more fair in terms of Wikipedian norms.  
% \end{description}

% \end{frame}


% \begin{frame}\frametitle{Recent changes filters on Wikipedia}
% \vspace{15em}
%     \centering
% \larger
% \begin{tikzpicture}

%   \node[anchor=west](flags) at (-7,2.7) {ORES Flags};
% %  \node[anchor=west](pagetitle) at (-4.7,2.7) {Page titles};
%   \node[anchor=west](userpagelink) at (-2.9,2.7) {Red user profile link};
%   \node[anchor=west](unregistered) at (1.4,2.7) {Unregistered editor};


%   \begin{scope}
%   \node[anchor=south, inner sep=0] (image) at (0,0) {\includegraphics[width=\textwidth]{resources/rcfilters_example_2.png}};
%   \draw [-stealth,ultra thick] (flags.220) -- ++(0,-0.4);
% %  \draw [-stealth,ultra thick] (pagetitle.south) -- ++(0,-0.4);
%   \draw [-stealth,ultra thick] (unregistered.200) -- (1.5,2);
%   \draw [-stealth,ultra thick] (userpagelink.280) -- (0.2,1.6);
% \end{scope}
% \end{tikzpicture}

% \raggedright

% %Flags are triggered by arbitrary thresholds, so we can use a regression discontinuity to estimate the causal effects of being flagged on moderator actions?

% \end{frame}


\begin{frame}\frametitle{RQ1: Flagging and over-profiling: User pages}

\begin{knitrout}
\definecolor{shadecolor}{rgb}{0.969, 0.969, 0.969}\color{fgcolor}
\includegraphics[width=\textwidth]{figures/knitr-h1_userpage_me_plot-1} 

\end{knitrout}

\end{frame}

% \begin{frame}\frametitle{RQ1: Flagging and over-profiling: User pages}

% \larger

% Actions by editors \emph{without} User pages are more sensitive \\ to flagging than are edits \emph{with} User pages at the \\ ``maybe damaging'' threshold.  

% %  There are few edits by editors with profile pages flagged at the ``likely damaging'' threshold.

%   This points to the opposite conclusion as when we looked at registration status: algorithmic flagging seems to amplify over-profiling of editors without User pages. 

% \end{frame}


\begin{frame}\frametitle{RQ2: Flagging and fair sanctioning: No User pages}

\larger

We don't detect a change in controversial sanctioning \\ for editors without user pages. 
  

\includegraphics[width=\textwidth]{figures/knitr-me_plot_H2_no_user_page-1} 

\end{frame}

\begin{frame}\frametitle{Conclusion}

\larger

  Perhaps editors without User pages are not over profiled, \\ or the algorithm is biased against them. \pause

  While our estimates of the effects of flagging are causal, \\ comparison between types of editors is not. \pause

  This is about moderation on Wikipedia, but social and psychological processes might be similar in high stakes settings. \pause

  While algorithmic flagging can improve fariness for some classes of user, the general relationship seems complex and contingent. \pause

\end{frame}

\begin{frame}

  Thank you!

  \href{mailto:nathante@uw.edu}{nathante@uw.edu}

  \includegraphics[height=1.5em]{figures/Twitter_Logo_Blue.pdf} @groceryheist

  Manuscript on Arxiv:  \url{https://arxiv.org/abs/2006.03121}

  \url{https://teblunthuis.cc}

  \url{https://communitydata.science}

\end{frame}

\appendix



\end{document}

%  LocalWords:  xcolor dvipsnames beamer ugm phv sep pageofpages px
%  LocalWords:  titleline alternativetitlepage titlepagelogo Torino
%  LocalWords:  watermarkheight watermarkheightmult bg makopurple fg
%  LocalWords:  noparskip colback colframe coltext coltitle fonttitle
%  LocalWords:  colorlinks linkcolor citecolor filecolor urlcolor
%  LocalWords:  unicode frametitle subbody subsubbody
